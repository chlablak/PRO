\documentclass[french]{article}
\usepackage[T1]{fontenc}
\usepackage[utf8]{inputenc}
\usepackage{lipsum}
\usepackage{lmodern}
\usepackage{geometry}
\usepackage{babel}
\usepackage{graphicx}
\usepackage{lastpage}
\usepackage{ragged2e}

\geometry{
 	a4paper,
 	total={210mm,297mm},
 	left=20mm,
 	right=20mm,
 	top=20mm,
 	bottom=20mm,
}

\usepackage{fancyhdr}
\pagestyle{fancy}

\lhead{Browne, Champion, Djomo,\\Hardy, Richoz, Rochat}
\chead{}
\rhead{26.02.2016}
\renewcommand{\headrulewidth}{0.4pt}
\renewcommand{\footrulewidth}{0.4pt}
	 
\begin{document}
	\centering
	\large{\textbf{PRO: Application de résolution de graphes}}
	
	\justify
	
	\section{Concept}
		Suite au cours de GRE, l'idée nous est venu de créer une application dédiée à la résolution des problèmes classiques de graphes. Celle-ci permet à l'utilisateur de saisir des sommets et des arrêtes, puis après sélection du problème à résoudre, l'application rend une solution textuelle et éventuellement graphique.
		
	\section{Spécifications}
		\subsection{Saisie du graphe}
			Le graphe doit être saisi au moyen d'un langage simple spécifique à l'application. Il sera donc nécessaire de \textit{parser} les commandes afin de générer le graphe. Celles-ci pourront également être sauvées dans un fichier texte, afin de pouvoir reprendre son travail d'une session à la suivante.
		
		\subsection{Interface graphique}
			La seconde partie de l'application sera dédiée à la représentation graphique. Les sommets pourront être déplacés au moyen du curseur pour pouvoir mieux visualiser la résolution.
			
		\subsection{Algorithmes sur les graphes}
			L'application permettra d'appliquer les algorithmes classiques des graphes, et sera facilement extensible à de nouveaux algorithmes. 
		
		\subsection{Exportation de graphe}
			La représentation graphique doit pouvoir être exporté dans un ficher au format SVG (scalable vector graphics).
		
	\section{Fonctions optionnelles}
		\subsection{Résolution du graphe automatisée}
			Selon le problème à résoudre, l'application déterminera automatiquement l'algorithme à utiliser et l'appliquera.
		
		\subsection{Interface graphique avancée}
			Le graphe généré sera automatiquement disposé de manière simple à lire (de façon planaire lorsque cela est possible), indépendamment de l'ordre de saisie des sommets. 
		
		\subsection{Parallélisation des traitements}
			Lorsque cela est possible, soit pour générer un résultat, soit pour certains algorithmes, l'application utilisera de la programmation concurrente afin d'accélérer les différents traitements.
			
\end{document}
