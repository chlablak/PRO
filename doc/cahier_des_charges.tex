\documentclass[french]{article}
\usepackage[T1]{fontenc}
\usepackage[utf8]{inputenc}
\usepackage{lipsum}
\usepackage{lmodern}
\usepackage{geometry}
\usepackage{babel}
\usepackage{graphicx}
\usepackage{lastpage}
\usepackage{ragged2e}
\usepackage{enumitem}

\geometry{
 	a4paper,
 	total={210mm,297mm},
 	left=20mm,
 	right=20mm,
 	top=20mm,
 	bottom=20mm,
}

\usepackage{fancyhdr}
\pagestyle{fancy}
\setlist[enumerate,1]{leftmargin=2cm}

\lhead{Browne, Champion, Djomo,\\Hardy, Richoz, Rochat}
\chead{}
\rhead{07.03.2016}
\renewcommand{\headrulewidth}{0.4pt}
\renewcommand{\footrulewidth}{0.4pt}
	 
\begin{document}
	\centering
	\large{\textbf{PRO: cahier des charges}}
	
	\justify
	
	\section{Spécification}
		L'objectif global de ce projet est la réalisation d'une application de résolution de problèmes classiques des graphes dans le cadre du cours PRO. Elle sera réalisée en Qt afin de pouvoir être déployée sur les plateformes Linux, Mac OS et Windows mais ne sera que testée sur cette dernière.
		
		\subsection{Maquette (pas définitive)}
			\includegraphics[width=0.9\textwidth]{maquette}
		
	\section{Fonctionalités}
		Les fonctionnalités suivantes devront être réalisées avant le 30 mai (semaine 14 du semestre). Les optionnelles seront à faire si le temps le permet, et les futures sont des améliorations de l'application non prévues dans le cadre du projet.
		\begin{enumerate}
			\item Saisie de graphes orientés ou non et pondérés ou non au moyen d'un langage spécifique.
			\begin{enumerate}
				\item Chargement d'un (grand) graphe depuis un fichier externe.
			\end{enumerate}
			
			\item Appliquer les algorithmes classiques aux graphes et afficher leurs résultats.
			\begin{enumerate}
				\item Les commandes peuvent être sauvées vers et chargées depuis un fichier texte.
				\item Plusieurs onglets peuvent être ouverts en parallèle.
				\item (optionnelle) Auto-complétion des fonctions et des variables pendant une saisie.
			\end{enumerate}
			
			\item Dessiner la représentation graphique d'un graphe défini précédemment.
			\begin{enumerate}
				\item Édition à la souris de la représentation graphique, reprise sur tous les dessins.
				\item Exportation des dessins au format SVG.
				\item (optionnelle) Exportation dans d'autres formats d'image.
				\item (future) Dessiner le graphe directement de manière optimale (avec le moins de chevauchement au niveau des arcs/arêtes).
			\end{enumerate}
			
			\item Un menu d'aide contenant les différentes entrées/fonctions possibles avec une documentation associée, ainsi qu'une option de recherche.
			\begin{enumerate}
				\item (future) Possibilité de lancer un algorithme via une fenêtre de dialogue.
			\end{enumerate}
		
			\item (...)
		\end{enumerate}
		
		\section{Planification}
			\subsection{Liste des tâches}
				(listes des tâches avec leur durée estimée et leur dépendance)
			\subsection{Graphe potientiels-tâches}
				(éventuellement)
			\subsection{Diagramme de Gantt}
				(avec prise en compte des retards)
			
\end{document}
