\documentclass[french]{article}
\usepackage[T1]{fontenc}
\usepackage[utf8]{inputenc}
\usepackage{lipsum}
\usepackage{lmodern}
\usepackage{geometry}
\usepackage{babel}
\usepackage{graphicx}
\usepackage{lastpage}
\usepackage{ragged2e}
\usepackage{enumitem}
\usepackage[normalem]{ulem}
\usepackage{longtable}

\geometry{
	a4paper,
	total={210mm,297mm},
	left=20mm,
	right=20mm,
	top=20mm,
	bottom=20mm,
}

\usepackage{fancyhdr}
\pagestyle{fancy}
\setlist[enumerate,1]{leftmargin=2cm}

\lhead{Browne, Champion, Djomo,\\Hardy, Richoz, Rochat}
\chead{}
\rhead{}
\renewcommand{\headrulewidth}{0.4pt}
\renewcommand{\footrulewidth}{0.4pt}

\begin{document}
	\centering
	\large{\textbf{PRO: Journal de travail}}
	
	\justify
	
	\section*{\huge \textit{Exemple}}
	\begin{longtable}{p{0.15\textwidth}|p{0.08\textwidth}|p{0.25\textwidth}|p{0.4\textwidth}}
		Date&Heures&Tâche&Description\\
		\hline
		01.01.2016&2.5&Implémentation class FooBar()&Implémentation des fonctions principales de la classe FooBar, tel que \textit{Lorem ipsum dolor sit}.\\
		\hline
		02.01.2016&1&Documentation&Description de quelque chose\\
		\hline
		02.01.2016&1.5&etc...&etc...
	\end{longtable}
	
	\section*{Christopher Browne}
	\begin{longtable}{p{0.11\textwidth}|p{0.07\textwidth}|p{0.3\textwidth}|p{0.4\textwidth}}
		Date&Heures&Tâche&Description\\
		\hline \hline
		21.03.2016 & 1.0 & Modélisation de la classe Graphe & Propositions avec le groupe \\
		21.03.2016 & 0.5 & Organisation groupe & \\
		21.03.2016 & 0.5 & Relecture & Cadre de développement, normalisation du langage, analyse des types et des opérations \\
		23.03.2016 & 1.0 & Étude des widgets Qt & \\
		23.03.2016 & 2.0 & Conception interface aide & \\
		\textbf{Total} & \textbf{5.0} &&\\
		\hline
		04.04.2016 & 0.5 & Documentation & Structure de base du rapport \\
		04.04.2016 & 1.5 & Implémentation interface aide & Création de la GUI pour l'aide \\
		06.04.2016 & 0.5 & Relecture & Grammaire EBNF, Conception de l'interpréteur\\
		08.04.2016 & 1.0 & Relecture tout & Suite à un conflit git\\
		08.04.2016 & 2.5 & Implémentation interface aide & Navigation \\
		\textbf{Total} & \textbf{6.0} &&\\
		\hline
		11.04.2016 & 3.0 & Implémentation interface aide & Navigation \\
		11.04.2016 & 0.5 & Relecture & \\
		15.04.2015 & 1.0 & Implémentation de la navigation & Départ à 0: utilisation QTextBrowser.\\
		\textbf{Total} & \textbf{4.5} &&\\
		\hline
		18.04.2016 & 1.0 & Implémentation interface aide & Ajout barre de recherche et fenêtre principale \\
		18.04.2016 & 1.0 & Conception pages aide & Template HTML\\
		22.04.2016 & 2.5 & Implémentation recherche & \\
		23.04.2016 & 2.0 & Implémentation recherche & \\
		23.04.2016 & 2.0 & Préparation présentation & Discussion groupe + powerpoint\\
		\textbf{Total} & \textbf{8.5} &&\\
		\hline
		25.04.2016 & 0.5 & Discussion groupe & \\
		25.04.2016 & 0.5 & Présentation intermédiaire & \\
		25.04.2016 & 2.0 & Conception dictionnaires & Recherche de la structure adéquate\\
		29.04.2016 & 2.0 & Implémentation dictionnaire & Sous forme de Ternary search Try\\
		\textbf{Total} & \textbf{5.0} &&\\
		\hline
		02.05.2016 & 1.5 & Implémentation dictionnaire & \\
		04.05.2016 & 1.0 & Implémentation dictionnaire & \\
		06.05.2016 & 1.0 & Tests dictionnaire & \\
		06.05.2016 & 1.0 & Corrections dictionnaire & \\
		08.05.2016 & 1.0 & Relecture rapport & \\
		\textbf{Total} & \textbf{5.5} &&\\
		\hline
		09.05.2016 & 2.5 & Implémentation interface aide & Ajout du volet de navigation \\
		13.05.2016 & 2.5 & Rédaction aide utilisateur & \\
		15.05.2016 & 4.5 & Implémentation auto-complétion & Adaptation QCompleter pour le QTextBrowser\\
		\textbf{Total} & \textbf{9.5} &&\\
		\hline
		16.05.2016 & 4.0 & Implémentation auto-complétion & \\
		18.05.2016 & 2.0 & Rédaction aide utilisateur & \\
		18.05.2016 & 1.5 & Documentation & section Interface\\
		20.05.2016 & 3.5 & Rédaction aide utilisateur & \\
		\textbf{Total} & \textbf{11.0} &&\\
		\hline
		23.05.2016 & 1.5 & Aide intégration GUI & \\
		23.05.2016 & 1.5 & Tests GUI/application & Recherche et documentation de bugs\\
		23.05.2016 & 1.5 & Correction auto-complétion & \\
		25.05.2016 & 5.5 & Correction de bugs & Aide utilisateur: problème lié aux signaux/slots\\
		26.05.2016 & 2.5 & Recherche/corrections bugs & \\
		27.05.2016 & 1.5 & Traduction aide utilisateur & \\
		27.05.2016 & 2.5 & Relecture rapport & \\
		29.05.2016 & 2.5 & Traduction aide utilisateur & \\
		29.05.2016 & 5.0 & Mise en commun et documentation & \\
		\textbf{Total} & \textbf{24.0} &&\\
		\hline
		30.05.2016 & 1.5 & Rendu & Impression, reliure... \\
		\textbf{Total} & \textbf{1.5} &&\\
		\hline
		\hline
		\textbf{Total} & \textbf{80.5} &&\\	
	\end{longtable}
	
	\section*{Patrick Champion}
	\begin{longtable}{p{0.11\textwidth}|p{0.07\textwidth}|p{0.3\textwidth}|p{0.4\textwidth}}
		Date & Heures & Tâche & Description\\ 
		\hline \hline
		18.03.2016 & 3.0 & Normalisation du langage & Début de l'analyse\\ 
		\textbf{Total} & \textbf{3.0} &&\\
		\hline
		21.03.2016 & 1.5 & Normalisation du langage & Suite de l'analyse\\
		23.03.2016 & 1.5 & Normalisation du langage & Suite de l'analyse\\
		24.03.2016 & 2.0 & Normalisation du langage & Grammaire EBNF\\  
		\textbf{Total} & \textbf{5.0} &&\\
		\hline
		04.04.2016 & 2.5 & Conception de l'interpréteur & Diagramme de flux des données\\
		05.04.2016 & 1.5 & Conception de l'interpréteur & Flux des données et début de la mémoire virtuelle\\
		06.04.2016 & 2.0 & Conception de l'interpréteur & Conception des types dans la mémoire virtuelle (TDH)\\
		08.04.2016 & 1.0 & Conception de l'interpréteur & Conception des types dans la mémoire virtuelle (Edge et Vertex)\\   
		\textbf{Total} & \textbf{7.0} &&\\
		\hline
		14.04.2016 & 3.0 & Conception de l'interpréteur & Tables des variables et des fonctions\\
		15.04.2016 & 1.5 & Conception de l'interpréteur & Transformation et arbre d'appels\\
		16.04.2016 & 1.5 & Conception de l'interpréteur & Modifications dans le rapport et interfaçage de fonction surchargée\\   
		\textbf{Total} & \textbf{6.0} &&\\
		\hline
		18.04.2016 & 2.0 & Conception de l'interpréteur & Fonctions surchargées, vérification des types et début de l'architecture\\
		19.04.2016 & 1.5 & Conception de l'interpréteur & Fin de l'architecture, diagramme de classes\\ 
		20.04.2016 & 2.0 & Implémentation & Début du parseur\\
		21.04.2016 & 1.0 & Implémentation & Grammaire en cours\\
		22.04.2016 & 1.0 & Implémentation & Grammaire presque finie, à tester\\
		23.04.2016 & 2.0 & Implémentation & Grammaire finie, mais à tester et corriger\\
		24.04.2016 & 1.5 & Implémentation & Grammaire ok mais améliorable, édition du rapport\\   
		\textbf{Total} & \textbf{11.0} &&\\
		\hline
		25.04.2016 & 0.5 & Présentation & Présentation intermédiaire de l'état du projet\\
		25.04.2016 & 1.0 & Implémentation & Début de la classe Optional\\
		26.04.2016 & 3.0 & Implémentation & Types ok (Array, Edge, Vertex, Number), VariableTable (manque Graph et TSTMap)\\
		27.04.2016 & 2.0 & Implémentation & FunctionCaller ok, FunctionTable commencée\\
		28.04.2016 & 1.5 & Implémentation & Preprocessor ok, FunctionTable presque finie\\
		29.04.2016 & 1.0 & Implémentation & FunctionTable finie, à tester\\  
		\textbf{Total} & \textbf{9.0} &&\\
		\hline
		02.05.2016 & 1.5 & Implémentation & ProcessingUnit commencée\\
		03.05.2016 & 2.5 & Implémentation & ProcessingUnit finie, à tester\\
		04.05.2016 & 2.0 & Implémentation & Début des fonctions 'basics'\\  
		\textbf{Total} & \textbf{6.0} &&\\
		\hline 
		09.05.2016 & 2.0 & Implémentation & Debugging et fonctions interfaçées\\
		10.05.2016 & 1.5 & Implémentation & Classes Data et Interpreter\\
		11.05.2016 & 0.5 & Discussion & Voir comment on créer des IGraph\\
		12.05.2016 & 1.5 & Implémentation & GraphWrapper et correction de la grammaire (+=, -=)\\
		13.05.2016 & 0.5 & Implémentation & basics pour les graphes, à impl.\\  
		\textbf{Total} & \textbf{6.0} &&\\
		\hline
		19.05.2016 & 3.0 & Implémentation & IGraph* et Vertex insertion\\
		20.05.2016 & 1.0 & Implémentation & toString générique\\  
		\textbf{Total} & \textbf{4.0} &&\\
		\hline
		23.05.2016 & 6.0 & Implémentation & GraphWrapper ok, load/save, interfaçage des algos, utilisation dans la GUI\\
		24.05.2016 & 3.0 & Implémentation & Correction de bugs, ajout de la génération aléatoire de graphe, interfaçage\\
		25.05.2016 & 5.0 & Implémentation & Création d'un timer (profiling), performances générales améliorées, interfaçags, ajout de l'export SVG dans la GUI, correction de bugs\\
		26.05.2016 & 5.0 & Implémentation & (Dé-)Sérialisation des données, interfaçage, mise à jour du rapport, performances améliorées (évite certaines copies inutiles), ajout d'un fonction size() pour les graphes et les arrays\\
		27.05.2016 & 3.0 & Implémentation & Correction de la fonction size(), insertions dans GraphWrapper améliorées, correctif sur draw() pour garder la même représentation\\
		28.05.2016 & 1.0 & Documentation & Amélioration du JT, mise à jour du rapport\\
		29.05.2016 & 6.0 & Documentation & Mise à jour du rapport (finale), préparation du déploiement de l'application, derniers tests, relecture, ...\\  
		\textbf{Total} & \textbf{29.0} &&\\
		\hline
		30.05.2016 & 2.0 & Rendu & Impression, reliure, ...\\
		\textbf{Total} & \textbf{2.0} && \\
		\hline
		\hline
		\textbf{Total} & \textbf{88.0} &&\\
	\end{longtable}
	
	\section*{Patrick Djomo}
	\begin{longtable}{p{0.15\textwidth}|p{0.08\textwidth}|p{0.25\textwidth}|p{0.4\textwidth}}
		Date&Heures&Tâche&Description\\
		\hline
	\end{longtable}
	
	\section*{Alain Hardy}
	\begin{longtable}{p{0.15\textwidth}|p{0.08\textwidth}|p{0.25\textwidth}|p{0.4\textwidth}}
		Date&Heures&Tâche&Description\\
		\hline\hline
		22.03.2016 & 1.0 & Apprentissage de QtDesigner & Apprentissage et étude des composants mis à disposition par QtDesigner.\\
		23.03.2016 & 1.0 & Apprentissage de QtDesigner & Apprentissage et étude des composants mis à disposition par QtDesigner.\\
		26.03.2016 & 2.0 & Apprentissage de QtDesigner & Apprentissage et étude des composants mis à disposition par QtDesigner.\\
		27.03.2016 & 2.0 & Apprentissage de QtDesigner & Apprentissage et étude des composants mis à disposition par QtDesigner.\\
		\textbf{Total} & \textbf{6.0} &&\\
		\hline
		04.04.2016 & 1.0 & Apprentissage de QtDesigner & Apprentissage et étude des composants mis à disposition par QtDesigner.\\
		05.04.2016 & 1.0 & Apprentissage de QtDesigner & Apprentissage et étude des composants mis à disposition par QtDesigner.\\
		07.04.2016 & 1.0 & Modélisation de l'interface graphique & Réalisation d'une maquette de l'interface graphique de l'application.\\
		09.04.2016 & 1.5 & Modélisation de l'interface graphique & Réalisation d'une maquette de l'interface graphique de l'application.\\
		10.04.2016 & 1.5 & Modélisation de l'interface graphique & Réalisation d'une maquette de l'interface graphique de l'application.\\
		\textbf{Total} & \textbf{6.0} &&\\
		\hline
		11.04.2016 & 1.5 & Création de l'interface graphique & Création de la fenêtre et des composants de base, telle que les onglets de fenêtres et les onglets pour les consoles multiples.\\
		12.04.2016 & 1.5 & Création de l'interface graphique & Début de la création d'une console personnalisée, mise en place de la structure de base.\\
		14.04.2016 & 1.0 & Création de l'interface graphique & Implémentation de la gestion des touches clavier dans la console.\\
		15.04.2016 & 1.0 & Création de l'interface graphique & Implémentation de la gestion des touches clavier dans la console.\\
		16.04.2016 & 2.0 & Création de l'interface graphique & Mise en place de la gestion du curseur afin d'empêcher l'utilisateur d'insérer du texte n'importe où.\\
		\textbf{Total} & \textbf{7.0} &&\\
		\hline
		19.04.2016 & 2.0 & Création de l'interface graphique & Fin de la gestion du curseur et implémentation du déplacement dans la commande via les flèches, et de la suppression de texte.\\
		21.04.2016 & 1.5 & Création de l'interface graphique & Début de l'implémentation de l'historique des commandes. Stockage des commandes lors de leur exécution.\\
		23.04.2016 & 1.5 & Création de l'interface graphique & Navigation dans l'historique des commandes et affichages dans la console de la commande restorée.\\
		24.04.2016 & 1.5 & Création de l'interface graphique & Implémentation de la création de nouvelle console, ainsi que de la fermeture.\\
		\textbf{Total} & \textbf{6.5} &&\\
		\hline
		25.04.2016 & 2.0 & Création de l'interface graphique & Mise en place d'un menu contextuel personnalisé. Création du widget et de ses composants.\\
		27.04.2016 & 2.0 & Création de l'interface graphique & Création des fonctions pour chacune des actions du menu contextuel.\\
		29.04.2016 & 2.0 & Création de l'interface graphique & Création d'une fenêtres permettant à l'utilisateur de saisir une texte. Sera utilisé pour la création d'une nouvelle console depuis la fenêtre principale.\\
		30.04.2016 & 1.5 & Création de l'interface graphique & Implémentation de la fonctionnalité de création de nouveaux onglets contenant des consoles.\\
		\textbf{Total} & \textbf{7.5} &&\\
		\hline
		02.05.2016 & 1.5 & Création de l'interface graphique & Mise en place de la fermeture des onglets.\\
		04.05.2016 & 2.0 & Modélisation de l'importation/exportation des sessions de travail & Réflexion sur l'implémentation de l'importation/exportation des sessions de travail. Les sessions de travails étant une sauvegarde toutes les console présentes dans l'applications, il faudrait en premier lieu implémentés la sauvegarde individuelle de console.\\
		06.05.2016 & 2.0 & Modélisation de l'importation/exportation des sessions de travail & Réflexion sur l'implémentation de l'importation/exportation des sessions de travail. Les sessions de travails étant une sauvegarde toutes les console présentes dans l'applications, il faudrait en premier lieu implémentés la sauvegarde individuelle de console.\\
		06.05.2016 & 2.0 & Modélisation de l'importation/exportation des sessions de travail & Réflexion sur l'implémentation de sauvegarde de console.\\
		\textbf{Total} & \textbf{7.5} &&\\
		\hline
		09.05.2016 & 2.0 & Implémentation de l'importation/exportation des sessions de travail & Implémentation de la sauvegarde de console dans un fichier.\\
		11.05.2016 & 1.5 & Implémentation de l'importation/exportation des sessions de travail & Fin de l'implémentation de la sauvegarde de console dans un fichier.\\
		13.05.2016 & 1.5 & Implémentation de l'importation/exportation des sessions de travail & Implémentation de la restauration de console depuis un fichier.\\
		15.04.2016 & 1.5 & Implémentation de l'importation/exportation des sessions de travail & Finalisation de la restauration de console depuis un fichier.\\
		\textbf{Total} & \textbf{6.5} &&\\
		\hline
		17.05.2016 & 2.0 & Implémentation de l'importation/exportation des sessions de travail & Implémentation de la sauvegarde et restauration de console, à partir des fonctions individuelles des consoles.\\
		19.05.2016 & 1.0 & Tests de l'importation/exportation des sessions de travail & Test sur la sauvegarde et la restauration de console individuelles. Modification du code suite à la découverte de bugs.\\
		20.05.2016 & 1.0 & Tests de l'importation/exportation des sessions de travail & Test sur la sauvegarde et la restauration de session de travail.\\
		21.05.2016 & 1.0 & Tests de l'importation/exportation des sessions de travail & Test sur la sauvegarde et la restauration de session et de console individuelle, afin de voir que l'utilisation des deux fonctionne bien ensemble.\\
		22.05.2016 & 1.0 & Tests de l'importation/exportation des sessions de travail & Test sur la sauvegarde et la restauration de session et de console individuelle, afin de voir que l'utilisation des deux fonctionne bien ensemble.\\
		\textbf{Total} & \textbf{6} &&\\
		\hline
		23.05.2016 & 1.0 & Tests de l'importation/exportation des sessions de travail & Test sur la sauvegarde et la restauration de console individuelles. Modification du code suite à la découverte de bugs.\\
		23.05.2016 & 2.0 & Création de l'interface graphique & Création d'un composant permettant l'affichage de graphe.\\
		23.05.2016 & 2.0 & Dessin du graphe avec GUI & Possibilité de dessiner des graphes depuis l'interface graphique.\\
		23.05.2016 & 1.5 & Intégration de l'interpréteur & Intégration de l'interpréteur à la console afin d'exécuter les commandes et d'en recevoir les résultats.\\
		25.05.2016 & 1.5 & Intégration de l'aide utilisateur & Intégration de l'aide utilisateur à l'interface graphique.\\
		25.05.2016 & 2.0 & Exportation SVG avec la GUI & Exportation depuis une commande dans la console de graphe au format SVG.\\
		25.05.2016 & 2.0 & Création de l'interface graphique & Améliorations de l'utilisations de l'interface en intégrant des raccourcis clavier.\\
		26.05.2016 & 2.0 & Tests globaux de l'application & Test d'utilisation de l'applications, notamment la création de graphe, l'affichage, la sauvegarde/restauration de console/session.\\
		26.05.2016 & 2.0 & Documentation & Vérification du code et ajout de commentaires.\\
		29.05.2016 & 4.0 & Documentation & Rédaction de la documentation.\\
		\textbf{Total} & \textbf{20} &&\\
		\hline
		\hline
		\textbf{Total} & \textbf{73.0} &&\\
	\end{longtable}
	
	\section*{Sébastien Richoz}
	\begin{longtable}{p{0.15\textwidth}|p{0.08\textwidth}|p{0.25\textwidth}|p{0.4\textwidth}}
		Date&Heures&Tâche&Description\\
		\hline \hline
		25.03.2016 & 0.5 &Etude des pattern Strategy et Visitor&Découverte des patterns\\
		25.03.2016 & 0.75&Enumération des algos à traiter&Liste des algos à traiter, classés par catégorie\\
		25.03.2016 & 1.5 &Modélisation du diagramme UML&Première ébauche du diagramme en appliquant le pattern Visitor\\
		\hline
		04.04.2016 & 1.5 &Conception de la classe Graph&Différents types de graphes\\
		05.04.2016 & 1   &Conception de la classe Graph&Différents types de graphes\\
		\hline
		11.04.2016 & 1.5 &Conception de la classe Graph&Restructuration\\
		\hline
		14.04.2016 & 3   &Rapport&Rédaction du chapitre sur l'architecture des graphes\\
		\hline
		18.04.2016 & 1.5&Conception de la classe Graph&Sommets et edges, liste d'adjacence et factories\\
		24.04.2016 & 4   &Déclarations des classes et de leurs méthodes&Classes Graph, Vertex, Edge et interface IGraph\\
		24.04.2016 & 0.5&Conception du pattern Visitor&Application du pattern Visitor aux graphes\\
		\hline
		25.04.2016 & 0.5 & Présentation & Présentation intermédiaire de l'état du projet\\
		01.05.2016 & 0.5 & Réalisation & BFS non-orienté\\
		01.05.2016 & 5 & Réalisation & Corrections de la classe CommonGraph\\
		\hline
		02.05.2016 & 3 & Conception & Restructuration de la classe Graph avec Patrick\\
		06.05.2016 & 0.3 & Documentation Algorithmes & Explication sur choix du type de retour des algos\\
		06.05.2016 & 1 & Implémentation des algorithmes & DFS non-orienté\\
		\hline
		10.05.2016 & 1.5 & Implémentation classe Graph & Tentative de résolution d'une erreur de compilation\\
		11.05.2016 & 4 & Implémentation classe Graph & Finalisation de la classe Graph et tests unitaires de celle-ci \\
		12.05.2016 & 6 & Implémentation classe Graph & Fournir une interface IGraph non templatée \\
		14.05.2016 & 2 & Implémentations algorithmes & Organisation des classes \\
		15.05.2016 & 2 & Implémentations algorithmes & tests des algos BFS, DFS, et création d'une classe pour appeler les algos \\
		\hline
		18.05.2016 & 2 & Implémentations algorithmes & Kruskal et test\\
		18.05.2016 & 1 & Implémentations algorithmes & Prim et test\\
		19.05.2016 & 3 & Correctif & Correction de warnings de compilation\\
		19.05.2016 & 1 & Implémentations algorithmes & Visiteur pour transformer un graphe en un autre (toutes les combinaisons possibles)\\
		20.05.2016 & 2 & Classe graphe & Complétion des classes edges (Edge, DiEdge, FlowEdge)\\
		21.05.2016 & 2 & Classe graphe & Complétion des classes graphes (Graph, DiGraph, FlowGraph)\\
		21.05.2016 & 3 & Implémentations algorithmes & BFS et DFS pour tous types de graphe\\
		22.05.2016 & 4 & Implémentations algorithmes & Composante connexe, composante fortement connexe, Dijkstra, Bellman-Ford\\
		\hline
		23.05.2016 & 8 & Tests algorithmes & Correctifs amenés à différentes méthodes\\
		24.05.2016 & 2.5 & Tests algorithmes & Correction d'un bug sur Prim\\
		25.05.2015 & 6 & Implémentations algorithmes & Version orientés/non orientées des algos\\
		26.05.2015 & 7 & Implémentations algorithmes & Push-relabel, FFEK (non fonctionnels)\\
		27.05.2015 & 6 & Implémentations algorithmes & Push-relabel, FFEK (non fonctionnels)\\
		28.05.2015 & 3 & Classe graphe & Commentaires et refactoring\\
		28.05.2015 & 3 & Documentation & Mise à jour de la section Graphe\\
	\end{longtable}
	
	\section*{Damien Rochat}
	\begin{longtable}{p{0.15\textwidth}|p{0.08\textwidth}|p{0.25\textwidth}|p{0.4\textwidth}}
		Date & Heures & Tâche & Description \\
		\hline \hline
		21.03.2016 & 1.5 & Séance de groupe & Modélisation d'un graphe \\	
		           & 2 & Étude de Qt & Apprentissage du système de dessin (View, Scene, Item, etc.) \\
		\hline
		04.04.2016 & 1.5 & Étude de Qt & Apprentissage du système de dessin (View, Scene, Item, etc.) \\
		\hline
		08.04.2016 & 4 & Étude de Qt & Apprentissage et tests du système de dessin (View, Scene, Item, etc.) \\
		\hline
		11.04.2016 & 1.5 & Dessin du graphe & Implémentation du dessin des sommets \\
		\hline
		16.04.2016 & 3 & Dessin du graphe & Implémentation du dessin des sommets \\
		\hline
		18.04.2016 & 1.5 & Dessin du graphe & Implémentation du dessin des arcs et arêtes \\
		\hline
		21.04.2016 & 2 & Dessin du graphe & Implémentation du dessin des arcs et arêtes \\
		           & 1 & Dessin du graphe & Petite review du code actuel \\

		\hline
		25.04.2016 & 0.5 & Présentation & Présentation intermédiaire de l'état du projet \\
		           & 2 & Edition du graphe & Implémentation du drag\&drop des sommets \\
		\hline
		26.04.2016 & 3 & Edition du graphe & Implémentation du drag\&drop des sommets \\
		\hline
		02.03.2016 & 1.5 & Edition du graphe & Apprentissage des slots et signaux de QT \\		 
		           & 2 & Edition du graphe & Implémentation du drag\&drop des sommets \\
		\hline
		03.05.2016 & 1 & Edition du graphe & Implémentation du drag\&drop des sommets \\
		\hline
		10.06.2016 & 3 & Exportation SVG & Apprentissage de QT \\
		           & 2 & Exportation SVG & Implémentation de l'exportation des graphes en SVG \\
		\hline
		15.06.2016 & 4 & Revue & Refactorisation du code \\
				   & 2 & Dessin du graphe & Implémentation des flow graphes \\
		\hline
		17.06.2016 & 3 & Interface & Modification du code pour utiliser les classes graphes finales \\
		\hline
		19.06.2016 & 1 & Interface & Liaison avec la GUI \\
				   & 1 & Interface & Correction des bugs \\
		\hline
		21.06.2016 & 2 & Dessin du graphe & Correction de bugs, amélioration de l'implémentation des graphes \\
		\hline
		22.06.2016 & 5 & Réutilisation des vues & Implémentation du manageur des vues des graphes permettant la réutilisation de la même vue si la graphe n'a pas changé. Création d'une méthode de hashage des graphes. \\
		\hline
		23.06.2016 & 3 & Documentation & Création des diagrammes de classe et rédaction de la documentation \\
		\hline
		24.06.2016 & 2 & Documentation & Création des diagrammes de classe et rédaction de la documentation \\
		\hline
	\end{longtable}
	
\end{document}
