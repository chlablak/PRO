\documentclass[french]{article}
\usepackage[T1]{fontenc}
\usepackage[utf8]{inputenc}
\usepackage{lipsum}
\usepackage{lmodern}
\usepackage{geometry}
\usepackage{babel}
\usepackage{graphicx}
\usepackage{lastpage}
\usepackage{ragged2e}
\usepackage{enumitem}
\usepackage[normalem]{ulem}

\geometry{
	a4paper,
	total={210mm,297mm},
	left=20mm,
	right=20mm,
	top=20mm,
	bottom=20mm,
}

\usepackage{fancyhdr}
\pagestyle{fancy}
\setlist[enumerate,1]{leftmargin=2cm}

\lhead{Browne, Champion, Djomo,\\Hardy, Richoz, Rochat}
\chead{}
\rhead{}
\renewcommand{\headrulewidth}{0.4pt}
\renewcommand{\footrulewidth}{0.4pt}

\begin{document}
	\centering
	\large{\textbf{PRO: Journal de travail}}
	
	\justify
	
	\section*{\huge \textit{Exemple}}
	\begin{tabular}{p{0.15\textwidth}|p{0.08\textwidth}|p{0.25\textwidth}|p{0.4\textwidth}}
		Date&Heures&Tâche&Description\\
		\hline
		01.01.2016&2.5&Implémentation class FooBar()&Implémentation des fonctions principales de la classe FooBar, tel que \textit{Lorem ipsum dolor sit}.\\
		\hline
		02.01.2016&1&Documentation&Description de quelque chose\\
		\hline
		02.01.2016&1.5&etc...&etc...
	\end{tabular}
	
	\section*{Christopher Browne}
	\begin{tabular}{p{0.11\textwidth}|p{0.07\textwidth}|p{0.3\textwidth}|p{0.4\textwidth}}
		Date&Heures&Tâche&Description\\
		\hline \hline
		21.03.2016 & 1.0 & Modélisation de la classe Graphe & Propositions de modélisation avec le groupe \\
		21.03.2016 & 0.25 & Organisation groupe & \\
		21.03.2016 & 0.5 & Relecture & Cadre de développement, normalisation du langage, analyse des types et des opérations \\
		23.03.2016 & 0.5 & Étude des widgets Qt & \\
		04.04.2016 & 1.5 & Interface aide utilisateur & Création de la GUI pour l'aide \\
		06.04.2016 & 0.25 & Relecture & Grammaire EBNF, Conception de l'interpréteur\\
		08.04.2016 & 0.5 & Relecture tout & Suite à un conflit git\\
		11.04.2016 & 0.25 & Discussion groupe & \\
	\end{tabular}
	
	\section*{Patrick Champion}
	\begin{tabular}{p{0.11\textwidth}|p{0.07\textwidth}|p{0.3\textwidth}|p{0.4\textwidth}}
		Date & Heures & Tâche & Description\\ \hline \hline
		18.03.2016 & 1.5 & Normalisation du langage & Début de l'analyse\\
		21.03.2016 & 0.5 & Normalisation du langage & Suite de l'analyse\\
		23.03.2016 & 1.0 & Normalisation du langage & Suite de l'analyse\\
		24.03.2016 & 2.0 & Normalisation du langage & Grammaire EBNF\\ \hline
		04.04.2016 & 1.0 & Conception de l'interpréteur & Diagramme de flux des données\\
		05.04.2016 & 1.5 & Conception de l'interpréteur & Flux des données et début de la mémoire virtuelle\\
		06.04.2016 & 2.0 & Conception de l'interpréteur & Conception des types dans la mémoire virtuelle (TDH)\\
		08.04.2016 & 1.0 & Conception de l'interpréteur & Conception des types dans la mémoire virtuelle (Edge et Vertex)\\
	\end{tabular}
	
	\section*{Patrick Djomo}
	\begin{tabular}{p{0.15\textwidth}|p{0.08\textwidth}|p{0.25\textwidth}|p{0.4\textwidth}}
		Date&Heures&Tâche&Description\\
		\hline
	\end{tabular}
	
	\section*{Alain Hardy}
	\begin{tabular}{p{0.15\textwidth}|p{0.08\textwidth}|p{0.25\textwidth}|p{0.4\textwidth}}
		Date&Heures&Tâche&Description\\
		\hline
	\end{tabular}
	
	\section*{Sébastien Richoz}
	\begin{tabular}{p{0.15\textwidth}|p{0.08\textwidth}|p{0.25\textwidth}|p{0.4\textwidth}}
		Date&Heures&Tâche&Description\\
		\hline \hline
		25.03.2016&0.5&Etude des pattern Strategy et Visitor&Découverte des patterns\\
		\hline
		25.03.2016&0.75&Enumération des algos à traiter&Liste des algos à traiter, classés par catégorie\\
		\hline
		25.03.2016&1.5&Modélisation du diagramme UML&Première ébauche du diagramme en appliquant le pattern Visitor\\
	\end{tabular}
	
	\section*{Damien Rochat}
	\begin{tabular}{p{0.15\textwidth}|p{0.08\textwidth}|p{0.25\textwidth}|p{0.4\textwidth}}
		Date&Heures&Tâche&Description\\
		\hline
	\end{tabular}
	
\end{document}
