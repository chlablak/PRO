\documentclass[french]{article}
\usepackage[T1]{fontenc}
\usepackage[utf8]{inputenc}
\usepackage{lipsum}
\usepackage{lmodern}
\usepackage{geometry}
\usepackage{babel}
\usepackage{graphicx}
\usepackage{lastpage}
\usepackage{ragged2e}
\usepackage{enumitem}
\usepackage[normalem]{ulem}

\geometry{
 	a4paper,
 	total={210mm,297mm},
 	left=20mm,
 	right=20mm,
 	top=20mm,
 	bottom=20mm,
}

\usepackage{fancyhdr}
\pagestyle{fancy}
\setlist[enumerate,1]{leftmargin=2cm}

\lhead{Djomo, Richoz}
\chead{}
\rhead{\today}
\renewcommand{\headrulewidth}{0.4pt}
\renewcommand{\footrulewidth}{0.4pt}
	 
\begin{document}
	\centering
	\large{\textbf{PRO: Enumération des algorithmes à traiter}}
	
	\justify
		
	\section{Concernant le graphe en soi}
	\begin{itemize}
		\item Graphe simple ?
		\item Graphe r-régulier ? <=> tous ses sommets de degré r
		\item Graphe est connexe, fortement connexe ?
		\item Graphe vide ?
		\item G est-il une forêt, un arbre ?
		\item Graphe cyclique, acyclique ?
		\item Graphe eulérien ? Chaine, chemin, cycle, circuit eulérien ?
		\item Graphe est-il planaire ?
		\item 
		\item Arborescence et anti-arborescence de racine et d'anti-racine
	\end{itemize}
	
	\section{Exploration d'un graphe}
	\begin{itemize}
		\item Parcours en largeur BFS $O(n+m)$
		\item Parcours en profondeur DFS $O(n+m)$
	\end{itemize}
		\subsection{Recherche des composantes connexes}
		\begin{itemize}
			\item 
		\end{itemize}
		\subsection{Recherche des composantes fortement connexes }
		\begin{itemize}
			\item Tarjan $O(n+m)$
			\item Kosaraju $O(n+m)$
		\end{itemize}
	
	\section{Arbres et arborescences}
		\subsection{Arbres recouvrants de poids total min/max}
		\begin{itemize}
			\item Kruskal $O(m^2)$
			\item Prim $O(n^2)$
			\item Boruvka
		\end{itemize}
		\subsection{Arborescences recouvrantes de poids total min/max}
		\begin{itemize}
			\item Chu-Liu (racine fixée, rajouter sinon)
		\end{itemize}
		\subsection{Arborescences recouvrantes de section min/max}
		\begin{itemize}
			\item Prim-orienté
		\end{itemize}
	
	\section{Plus courts chemins}
		\subsection{Depuis une source unique}
		\begin{itemize}
			\item Bellman-Ford $O(n*m)$
			\item Dijksra $O(m + n*log(n))$
		\end{itemize}
		\subsection{Entre tous les couples de sommets}
		\begin{itemize}
			\item Floyd-Warshall $O(n^3)$
			\item Johnson $O(n^3)$
		\end{itemize}
		\subsection{Entre deux sommets}
		\begin{itemize}
			\item Dijkstra bidirectionnel
		\end{itemize}
		
	\section{Graphes sans circuit}
	\begin{itemize}
		\item Tri topologique $O(n+m)$
		\item Plus court chemin dans un réseau sans circuit
		\item Plus long chemin dans un réseau sans circuit
	\end{itemize}
	
	\section{Familles et problèmes classiques de la théorie des graphes}
	\begin{itemize}
		\item Couplage max dans un graphe biparti => voir Flots
		\item un recouvrement (toutes les arêtes sont touchées)
		\item un transversal (tous les sommets sont touchés)
	\end{itemize}
	
	\section{Graphes eulériens}
	\begin{itemize}
		\item Algorithme de construction d'un cycle/circuit eulérien
		\item (futur) Problème du postier chinois
	\end{itemize}
	
	\section{Flots}
	\begin{itemize}
		\item Flots de valeur max (algorithme de FFEK), avec capacité de la coupe $O(m^2 * n), O(n^5)$ pour un réseau dense.
		\item Flot de valeur max à cout min (algo Busacket-Gowen)
		\item Couplage max dans un graphe biparti
		\item (optionnel) Problème de transbordement
		\item (futur) Couplage max (de poids min)
	\end{itemize}
			
\end{document}
